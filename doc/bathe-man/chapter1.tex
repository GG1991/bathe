\chapter{Finite Elements for solids and structures}

\section{Control load methods}

\subsection{Newton-Raphson}

\subsection{Arc-length methods}

Basically the idea is to imposed a variable external load that depends on a parameter $\lambda$ with the form:

\begin{equation}
\overline{f}_{\text{ext}} = \lambda \hat{f}
\end{equation}

So the equibrium equations can be expressed as:

\begin{equation}
\overline{f}_{\text{int}}(\overline{u}) = \lambda \hat{f}
\end{equation}

And a constrain equation which serves to limit the evolution of the solution on the $(\overline{u},\lambda)$ space.

\begin{equation}
\varphi(\overline{u},\lambda) = 0
\end{equation}

Finally the systems of equations to solve is:

\begin{equation}
\begin{cases} 
\overline{f}_{\text{int}}(\overline{u}) - \lambda \hat{f} = 0\\ 
\varphi(\overline{u},\lambda) = 0 
\end{cases}
\end{equation}

that can be written in a matricial form as:

\begin{equation}
  \overline{R}(\overline{u},\lambda) =
  \begin{bmatrix}
  \overline{f}_{\text{int}}(\overline{u}) - \lambda \hat{f} \\ 
  \varphi(\overline{u},\lambda) 
  \end{bmatrix}
  = \overline{0}
\label{arclen_sys_eq}
\end{equation}

Applying a multidimensional Taylor expanssion to the expression \ref{arclen_sys_eq}:

\begin{equation}
  \overline{R}(\overline{u},\lambda) \approx
  \overline{R}(\overline{u}_n,\lambda_n)
  +
  \begin{bmatrix}
  \frac{\partial \overline{f}_{\text{int}}}{\partial\overline{u}} & -\hat{f} \\ 
  \frac{\partial \varphi}{\partial\overline{u}} & \frac{\partial \varphi}{\partial\lambda} \\ 
  \end{bmatrix}
  \cdot
  \begin{bmatrix}
  \Delta u^i \\ 
  \Delta \lambda^i 
  \end{bmatrix}
\end{equation}

Where we want:

\begin{equation}
  \overline{R}(\overline{u},\lambda) = 0
\end{equation}

So the increments $ \Delta u^i$ and $\Delta \lambda^i$ can be calculated by an iterative procedure as:

\begin{equation}
  \begin{bmatrix}
  \Delta u^i \\ 
  \Delta \lambda^i 
  \end{bmatrix}
  =
  \begin{bmatrix}
  \frac{\partial \overline{f}_{\text{int}}}{\partial\overline{u}} & -\hat{f} \\ 
  \frac{\partial \varphi}{\partial\overline{u}} & \frac{\partial \varphi}{\partial\lambda} \\ 
  \end{bmatrix}^{-1}
  \cdot
  \overline{R}(\overline{u}_n^{i-1},\lambda_n^{i-1})
\end{equation}

and then simply:

\begin{equation}
\begin{cases} 
  u^i = u^{i-1} + \Delta u^i\\
  \lambda^i = \lambda^{i-1} + \Delta \lambda^i
\end{cases} 
\end{equation}




The secret of the method is to determine a suitable function $\varphi(\overline{u},\lambda)$ according to the problem we
want to solve.

\subsubsection{Arc-Length method based on energy.}

This method is similar to the one proposed on \cite{arclength_borst}. The basic idea is to select a
$\varphi(\overline{u},\lambda)$  that updates at every time step and imposed a a constrain in energy accumulation or
dissipation according to the phase of the problem.. 

During a time step where no energy is dissipated, for example in a pure elastic situation the elastic energy acumulated
can be expressed as:

\begin{equation}
U = \frac{1}{2}\left( \lambda \overline{u}^T - \lambda_n \overline{u}_{n}^T \right) = \Delta \tau^U
\end{equation}

so the constrain equation can be fix as :

\begin{equation}
\varphi^U(\overline{u},\lambda) = 
\frac{1}{2}\left( \lambda \overline{u}^T - \lambda_n \overline{u}_{n}^T \right) - \Delta
\tau^U = 0
\end{equation}

On the other hand on a dissipative situation, the energy lost by the system can be expressed as:

\begin{equation}
D = \frac{1}{2}\left( \lambda_n \overline{u}^T - \lambda \overline{u}_{n}^T \right) = \Delta \tau^D
\end{equation}

and the constrain can be written as:

\begin{equation}
\varphi^D(\overline{u},\lambda) = 
\frac{1}{2}\left( \lambda_n \overline{u}^T - \lambda \overline{u}_{n}^T \right) - \Delta
\tau^D = 0
\end{equation}
